\documentclass[12pt,a4paper,ngerman]{report}
\usepackage{babel}
%\usepackage{natbib}
\usepackage{url}
%\usepackage[left=2cm, right=1.5cm, top=2cm, bottom=2cm]{geometry}
%\usepackage[ansinew]{inputenc}
\usepackage{amsmath}
\usepackage{nicefrac} % macht schöne Brüche mit querstrich mit \nicefrac{1}{2}
\usepackage{graphicx}
%\graphicspath{}
\usepackage{titlesec}% um chapterüberschriften anzupassen.
\titleformat{\chapter}{\normalfont\huge\bf}{\thechapter.}{20pt}{\huge\bf}
\usepackage{parskip}
\usepackage{fancyhdr}
\usepackage{amsfonts}
\usepackage{float}
\usepackage{caption}
\usepackage{subcaption} % for \begin{subfigure}
	
\usepackage{csquotes} % mit \enquote{blabla} tolle anfürungsstriche erstellen
%\usepackage{physics} %lässt mich \bra und \ket benuzen %im konflict mit siunitx

\usepackage{pgfplots} %für plots
\pgfplotsset{compat=newest}

\usepackage{varioref} % macht mit \vref{} viel bessere referenzen
\usepackage{hyperref} % macht klickbare referenzen

\usepackage{xcolor, soul} %mit \hl{} kann man toll Sachen hervorheben.
\newcommand{\highlight}[1]{%
	\colorbox{yellow!50}{$\displaystyle#1$}} % mit \highlight{} kann man sogar in Gleichungen hervorheben

\usepackage{vmargin}
\usepackage[section]{placeins}
\usepackage{capt-of}
\usepackage{enumitem}
\usepackage{multirow}
\usepackage{blindtext}
\usepackage[version=4]{mhchem} % um Chemische Elementsymbole zu benutzen: \ce{H20}

\usepackage{pdfpages} % um PDFs einzufügen

%spread to latex:
\usepackage{booktabs, multirow} % for borders and merged ranges
\usepackage{changepage,threeparttable} % for wide tables

\providecommand{\e}[1]{\ensuremath{\cdot 10^{#1}}}
\providecommand{\fehlt}{\textcolor{red}{{ ¡Fehlt! }}}

\providecommand{\versuchstitel}{Perowskitsolarzellen} % Hier den Versuchstitel eintragen!

\usepackage{siunitx}
\sisetup{
	separate-uncertainty = true,
	%per-mode = fraction,
	%per-mode = symbol
}
\DeclareSIUnit\bar{bar}
\DeclareSIUnit\atomicmassunit{u}
\usepackage{isotope}


\setmarginsrb{3 cm}{2.5 cm}{3 cm}{2.5 cm}{1 cm}{1.5 cm}{1 cm}{1.5 cm}
\title{\versuchstitel}


\author{Frederik Uhlemann, Florian Adamczyk}
% Author
\date{\today}
% Date

\makeatletter
\let\thetitle\@title
\let\theauthor\@author
\let\thedate\@date
\makeatother

\pagestyle{fancy}
\fancyhf{}
\rhead{\theauthor}
\lhead{\versuchstitel}
\cfoot{\thepage}
%%%%%%%%%%%%%%%%%%%%%%%%%%%%%%%%%%%%%%%%%%%%
\begin{document}
		
	%%%%%%%%%%%%%%%%%%%%%%%%%%%%%%%%%%%%%%%%%%%%%%%%%%%%%%%%%%%%%%%%%%%%%%%%%%%%%%%%%%%%%%%%%
	
	\begin{titlepage}
		\centering
		\vspace*{0.5 cm}
		% \begin{large} Justus-Liebig-Universität\\ Gießen \end{large}
		\includegraphics[width = 0.6 \textwidth]{JLU_Giessen-Logo}	%University Logo
		\\[2.0 cm]
		% \begin{center}    \textsc{\Large Justus - Liebig - Universität}\\{Giessen}\\[0.8cm]	\end{center}% University Name
		Versuch 7 des\\
		\textsc{\Large Fortgeschrittenen-Praktikums}\\ [0.3 cm]				% Course Code
		\rule{\linewidth}{0.2 mm} \\[0.4 cm]
		{ \huge \bfseries \thetitle}\\%%% TITEL HERE
		\rule{\linewidth}{0.2 mm}\\
		Versuchstermin 21.6.2024 \\
		~ \\
		[2.0 cm]
		
		
		\begin{minipage}{0.49\textwidth}
			\begin{flushleft}
				 \emph{Praktikumsbetreuer:}\\
				 M.Sc. Tim P. Schneider\\
				 %  Affiliation\\
				 \small{\href{mailto:tim.schneider@ap.physik.uni-giessen.de}{tim.schneider@ap.physik.uni-giessen.de}}
			\end{flushleft}
		\end{minipage}~
		\begin{minipage}{0.49\textwidth}
			\begin{flushright}
				\emph{Protokoll von:} \\
				
				\large{Frederik Uhlemann}\\
				\small{\href{mailto:frederik-vincent.uhlemann@physik.uni-giessen.de}{frederik-vincent.uhlemann@physik.uni-giessen.de}\\~\\
					%Matrikel Nr.: \:  \\[0.5cm]
					%\href{mailto:}{}
				}
				\large{Florian Adamczyk} \\
				\small{\href{mailto:florian.marius.adamczyk@physik.uni-giessen.de}{florian.marius.adamczyk@physik.uni-giessen.de}\\
					%Matrikel Nr.: \: 8105234}
			}
		\end{flushright}
	\end{minipage}
	
	\end{titlepage}
	
%%%%%%%%%%%%%%%%%%%%%%%%%%%%%%%%%%%%%%%%%%%%%%%%%%%%%%%%%%%%%%%%%%%%%%%%%%%%%%%%%%%%%%%%%
\setcounter{secnumdepth}{3}
\setcounter{tocdepth}{4}
\tableofcontents
%\newpage

%%%%%%%%%%%%%%%%%%%%%%%%%%%%%%%%%%%%%%%%%%%%%%%%%%%%%%%%%%%%%%%%%%%%%%%%%%%%%%%%%%%%%%%%%
%\renewcommand{\thesection}{\arabic{section}} %lässt in den subsections die erste zahl von darüberliegenden chapter weg.

%\pagebreak
	
%\setcounter{chapter}{-1}
\chapter*{Einleitung}\addcontentsline{toc}{chapter}{Einleitung}
	\fehlt

	% notiz an mich: mit "~ bewirke ich einen geschützten bindestrich an dem nicht getrennt werden darf.
	% nur eine ~ macht ein geschütztes (normales) Leerzeichen. \, macht ein halbes geschütztes Leerzeichen.

\chapter{Theorie}
	\section{Kenngrößen von Solarzellen}
	Als wird folgend auf die Grundsätzlichen Eigenschaften von Solarzellen eingegangen, damit die Eigenschaften der hergestellten Perowskitsolarzelle eingeordnet werden können.\\
	Eine allgemeine Solarzelle verhält sich elektronisch ähnlich, wie eine Halbleiterdiode. In Abbildung \fehlt ist eine Solarzellenkennlinie abgebildet, der Unterschied zur Diodenkennlinie liegt allein in der Verschiebung in negative y-Richtung durch den Wert des Photostroms $j_{ph}$. Aus diesem Grund ist auch die Formel für die Stromcharakteristik bis auf diesen Wert identisch. 
	\begin{equation}
		j(V) =  j_S \left(\text{exp}\left(\frac{e V}{m k_B T}\right) - 1\right) - j_{ph}
	\end{equation}
	Der Photostrom kann für Perowskitsolarzellen folgendermaßen ausformuliert werden:
	\begin{equation}
		j_{ph} = e I_0 \eta_{CC} \eta_{inj} (1-\text{exp}(-\alpha d))
	\end{equation}
	Die dabei bisher aufgetretenen Variablen sind:
	\begin{itemize}
		\item Spannung $V$
		\item Sättigungsstrom $j_S$
		\item Elementarladung $e$ und Boltzmannkonstante $k_B$
		\item Idealitätsfaktor $m$
		\item Temperatur $T$
		\item Intensität des einfallenden Lichts $I_0$
		\item Effizienz für erzeugte Löcher/Elektronen transportierende Schichten zu erreichen $\eta_{CC}$
		\item Effizienz für Löcher/Elektronen in die Schichten einzutreten $\eta_{inj}$
		\item Absorptionskoeffizient $\alpha$
		\item Dicke der absorbierenden Schicht $d$
	\end{itemize}
	Nach dieser kurzen Einführung können die typischen Kenngrößen von Solarzellen erläutert werden.

	\paragraph{Kurzschlussstromdichte}
	\paragraph{Leerlaufspannung}
	\paragraph{Wirkungsgrad}
	\paragraph{Füllfaktor}
	\paragraph{externe Quanteneffizient}



	

\chapter{Aufbau und Durchführung}
	\fehlt
	
	


\chapter{Auswertung}
	\fehlt
\chapter{Fazit}
	\fehlt

\listoffigures%\addcontentsline{toc}{chapter}{\listfigurename}
	
\begin{thebibliography}{111}%\addcontentsline{toc}{chapter}{Literaturverzeichnis}
	\bibitem{Anleitung}
	Anleitung zum Fortgeschrittenen"~Praktikum.\\ \glqq Versuch: Organisch"~anorganisch hybride Perowskitsolarzellen\grqq.\\ Justus-Liebig-Universität Gießen. Sommersemester 2024.
	
	\bibitem{Perowskit}
		\fehlt
	\end{thebibliography}


\chapter*{Anhang}\label{ch:Anhang}\addcontentsline{toc}{chapter}{Anhang}
\FloatBarrier
	\begin{figure}[ht]
	\centering
%	\includegraphics[width=\textwidth]{data/Testat.pdf}		
	\caption[Testat]{Das Testat des Versuchs}
	\label{fig:Testat}
\end{figure}

\end{document}
